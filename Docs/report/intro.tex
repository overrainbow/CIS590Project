

\indent\indent Since the advent of video games, there has been a controversy surrounding this form of media and game content. Violence and gore are very prevalent in many of the most popular titles and, indeed, they are often a selling point for developers. Games are thought to be more influential and emotionally compelling than films, as the player interacts dynamically with the virtual world and essentially ``becomes" the character to guide their actions, rather than just watching actors on a screen. Again echoing the sentiment expressed in ``Digital Nation," this technology is moving so fast that we do not know the longterm effects that it will have on us and our minds, especially the developing minds of children.

Recent increases in gun-related violence initiated by young people has caused quite a stir and has lead people to suspect that there are some connections between such acts of violence and video game play. Varieties of studies of this have shown correlations between violent game play and hostile behaviors, violent responses, and even reduced school performance \cite{barlett2009, gentile2004, anderson2003}. At present, the causal relationship between violent video games and aggressive behaviors is still to be proven and further established due to the lack of supporting experiments and theory analysis \cite{erguson2013, ferguson2008,makuch2013,markey2014}. However, it is not hard to see the influences these video games can have on the current and future generations, such as causing degraded communication skills, morality, health, and even altering players' own personalities. Consequently, people can easily find these correlations, either directly or indirectly, between their hostile acts and these video games, regardless of proof of causation. As a result, many governmental policies have arisen in an attempt to restrict access of video games with mature content to underage individuals. Censorship is another measure taken to protect youth, and in some countries censorship is not a voluntary act.

Purchase restrictions and censorship are not a complete answer, however, as game developers feel their right to free speech is not being respected if their work is being altered or restricted in any way. Social norms of modern society, in which youth are gaining increased irreverence for law surrounding technology renders many laws useless. Norms further complicate the issue, as generally parents and guardians of children are not as savvy with technology and modern culture, and therefore are unaware of the negative content present in the games their children play. Lastly, enforcement of age limits on software is incredibly difficult. Therefore, public policy surrounding these games is an open issue that merits further investigation.

By definition as an ethical issue, there are positives and negatives to both sides of the argument. Censorship and restriction can greatly impact all stakeholders in different ways. Ethical analysis of the situation, using two workable theories, demonstrates the two sides of the debate. Using these theories as a base, we further explore each viewpoint with additional evidence and arguments.