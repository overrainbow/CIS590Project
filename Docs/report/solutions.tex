\indent\indent Based on our analysis, we found it very important for us to find solution to the issue of violent video games which protects the young generation from violent content but also encourages the game designers to create new work. The current policy in the USA focuses on a rating system and sale restriction. These two methods work jointly. The rating system categorizes games into different types of age-appropriateness according to the game's content. Distributors will be able to sell the product based on their rating. With the sale restriction, government regulates the selling behaviors of distributors so that they will not sell products to inappropriate users. Unfortunately, this is not a complete solution because age limits are ignored, and there is little or no enforcement of the sale restriction law. Therefore, new solutions need to be created. From our analysis we see that any regulation must make a balance between the interests of users and developers, so we have brainstormed several proposed solutions that aim to strike this balance. 

Law is the direct and powerful way of regulation. However, current laws enforcing censorship do not fully respect the creative work of game developers, as it forces modification of the game content and makes it inconsistent by breaking the story into several independent parts. For the developers, this modification deviates their original design of games and frustrates their motivation to develop new games. When the online game ``World of Warcraft" was introduced into China, the game was modified tremendously due to violent content. For instance, in this game, the color of blood was changed into green, which is even opposite to common sense. To balance the both sides of users and developers, a law could require game manufacturers to develop different versions for younger users, in which gore and violence are reduced. Another version of the game intended for adults could keep the integrity of the content intact and satisfy the creativity of the developers. 

Another possible solution is to build a better rating system that rates a game not only considering its content but also thoroughly considering other factors related to the use and distribution of the game. 

Government is able to play a critical role in leading the market into a good situation that all the software or game distributors could consciously restrict their selling behaviors. The government could set up incentive mechanisms to the merchants who stick on the rating system and only sell video game to the appropriate buyers. The incentive mechanism may include reputation improvement, tax breaks, or a monetary reward. Since the merchants would benefit from their self-discipline, they would most likely actively participate in popularizing the sales restriction.

Still, any restriction on sales is not the perfect solution, as a game could be sold legally, but given to an underage person to play. One way to possibly solve this would be to also require the developers to create a mechanism in the code to track the use of the game to ensure that a game containing violent content was used by a gamer of the proper age. In addition, it may be possible for the developers to add user authentication module to the game. When gamer is going to play the game, he or she will be required to authenticate identity which has to match purchase history of game. In this way, it keeps the integrity of game content and restricts underage users from playing violent games.

In addition to the efforts of developers to regulate users' behavior in architecture(code) part, it is necessary to improve parents' awareness about the correlation between the inappropriate video games and violent behaviors of children. Our government should enhance the visibility of such issues, and our community could also offer some workshops to parents to show them the importance of preventing children from experiencing games that contain violent material. When people realize the potential influence of games on children, more and more people will likely become more proactive in parenting and protecting children from violent games.

In conclusion, balancing the interests of developers and users can be a complicated mission. Since each proposed solution has its weakness, combining them together would achieve to a better performance. 

