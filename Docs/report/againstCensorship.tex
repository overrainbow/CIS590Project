%Outline:
%-Kantian analysis shows censorship is wrong, even if it does protect children
%- Censorship stifles free speech
%-Censorship in this way is wrong because there is no causation proven yet in the relationship between violence and video games.. 

\indent \indent As we have seen with several accounts of censorship, Kantian analysis shows censorship of video games to be wrong. We start by using the first formulation, the universal rule, to examine the act of censoring or restricting violent video games in any way. If every media company was subjected to censorship by the government, then media would not be unique and explore issues deemed ``inappropriate" by some third party. Governments would have the say as to what is ``unsuitable," but this may not reflect the ideas brought forth by the governed citizens. 

The second formulation of Kantian analysis inspects the extent to which an action uses people as a means to an end. By removing or altering pieces of a person or group's intellectual property, governments are imposing their views on the public and not respecting the artistic freedom of the IP's creators. Thus, citizens and creators both are being treated as means to an end. 

By both formulations, we see Kant would find such acts of video game censorship and restriction to be wrong. 

Muffling free speech is a large issue with censorship, and one that has come to the forefront as a result of attempts at restricting violent video games. Currently in Germany, due to sensitivities surrounding World War II, the German government is very protective of its people in regards to shielding them from media that could possibly encourage violent behavior. For the past several years, video games have been the major focus of this, causing the country to break away from the standard Pan European Game Information (PEGI) rating system \cite{pegi} and enact their own, more strict system called Unterhaltungssoftware Selbstkontrolle (USK) \cite{usk}. The German government requires that every game sold in Germany must be submitted to the USK to obtain a rating and games seen to be too realistically violent will be denied a rating at all. Games with a USK 18 rating cannot be advertised or sold on a shelf in stores. They must be held behind the counter and only purchased with valid IDs proving the buyer's age. Making customers have to explicitly ask for games that are considered ``inappropriate" is an attempt to alter social norms, making it taboo to even purchase them.

%Another good example of strict censorship attempts occurs in Germany. Due to sensitivities surrounding World War II, the German government is very protective of its people in regards to shielding from media that could possibly encourage violent behavior. The main form of media in question is video games.  where any game considered to be realistically violent is essentially banned. 

Crytek, a game company headquartered in Germany that specializes in violent shooter games, has had significant issues with the German government and is struggling to keep in business. The president of Cryteck, Cevat Yerli, stated how Germany's move stifled the creation of art: ``A ban on action games in Germany is concerning us because it is essentially like banning the German artists that create them. If the German creative community can't effectively participate in one of the most important cultural mediums of our future, we will be forced to relocate to other countries."

%http://www.pcgames.de/Crysis-PC-119380/News/Cevat-Yerli-ueber-die-so-genannten-Killerspiele-691719/2/