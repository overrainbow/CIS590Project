%Outline:
%-Kantian analysis shows censorship is wrong, even if it does protect children
%- Censorship stifles free speech
%- Censorship is against individual rights that the US strives to uphold
%-Censorship in this way is wrong because there is no causation proven yet in the relationship between violence and video games.. 

\indent \indent As we have seen with several accounts of censorship attempts such as protecting children online, Kantian analysis shows the censorship and restriction of video games to be wrong. We start by using the first formulation, the universal rule, to examine the act of censoring or restricting violent video games in any way. If every media company was subjected to censorship by the government, then media would not be unique and could not explore issues deemed ``inappropriate" by some third party. Governments would have the say as to what is ``unsuitable," but this may not reflect the ideas brought forth by the governed citizens. Furthermore, this would not be respecting the individual rights of the citizens, who should be free to watch, read, or play whatever they wish. Similarly, if every media was subject to restrictions laid down by the government in power, a tyrannical governmental force could essentially remove any media from circulation or impose tight restrictions on its purchase. This would render the creation of intellectual property practically useless.

The second formulation of Kantian analysis inspects the extent to which an action uses people as a means to an end. By removing or altering pieces of a person or group's intellectual property, governments are imposing their views on the public and not respecting the artistic freedom of the IP's creators. Restricting the sales of games to any party is condescending, treating people of the group as if they cannot know what is morally right or wrong. Thus in both accounts, citizens and creators both are being treated as means to an end. By both formulations, we see Kant would find such acts of video game censorship and restriction to be wrong. 

The ethical analysis provides the basis for the argument against censoring video games. There are several other arguments against this act, namely the suppression of free speech and individual rights, and the lack of studies providing actual proof that video games cause violence. %add more

Freedom of speech and freedom of artistic expression is highly valued in Western culture. The main argument against censorship of video games is the limitation of free speech that such an act imposes. By having elements of their game altered or sometimes even removed entirely, developers feel they are not being allowed artistic freedom and that their expression is being stifled. By not being able to produce what they desire, game developers also feel their audience and, by extension, their range of paying customers is being limited. There have been several recent examples of developers voicing their frustration by being inhibited as artists, where filmmakers and other media developers are not silenced in this way.

Currently in Germany, due to sensitivities surrounding World War II, the German government is very protective of its people in regards to shielding them from media that could possibly encourage violent behavior. For the past several years, video games have been the major focus of this, causing the country to break away from the standard Pan European Game Information (PEGI) rating system \cite{pegi} and enact their own, more strict system called Unterhaltungssoftware Selbstkontrolle (USK) \cite{usk}. Under this rating system, any game considered to be realistically violent is essentially banned. The German government requires that every game sold in Germany must be submitted to the USK to obtain a rating and games seen to be too realistically violent will be denied a rating and therefore not able to be sold in the country's borders. Games with a USK 18 rating, which can be equivalent to an ESRB Teen (T) or Mature (M) rating for violence, cannot be advertised or sold on a shelf in stores. They must be held behind the counter and only purchased with valid IDs proving the buyer's age. Making customers have to explicitly ask for games that are considered ``inappropriate" is an attempt to alter social norms surrounding these games, making it taboo to even purchase them.

Crytek, a game company headquartered in Germany that specializes in violent shooter games, has had significant issues with the German government and is struggling to remain in operation. The president of Crytek, Cevat Yerli, is very much against censorship and feels it stifles the creation of art and is blinding the people to the future of entertainment. When asked about the matter, he stated that ``a ban on action games in Germany is concerning us because it is essentially like banning the German artists that create them. If the German creative community can't effectively participate in one of the most important cultural mediums of our future, we will be forced to relocate to other countries." \cite{yerli}

%%%%Movies are not regulated in this way

In Western culture and ethics, the rights of the individual are often viewed as equally or sometimes even more important than the rights of the collective. The act of censoring is calling into question an individual's rights to view, read, watch, or play what they like. Similar to the recent gun law debates, in which people state that the Second Amendment protects an individual right to own property and that banning guns violates their rights, the anti-censorship side of the video game regulation debate argue that their rights and freedoms of expression as individuals are imposed upon if regulation is implemented. Censorship may be beneficial to the collective society as a whole, but many feel that it is not ethical to sacrifice their own freedoms for this profit. 

\hannah{I may add more here...}

The events inciting the move toward regulation and censorship have been the recent increase in gun-related violence and the studies following the incidents that have correlated such acts of violence with violent video game play. However, these studies have only been able to show correlations of the factors. Thus far, the causation of violent behavior has not been proven, causing many to believe that there is another factor involved in these cases of violence behavior that has not been inspected thoroughly \cite{ferguson2013, ferguson2008}. Former FBI profiler, Mary Ellen O'Toole, states that violent video games are not the sole cause factor causing violent acts, and are just one risk to individuals who are already contemplating acting out violently\cite{makuch2013}. Surprisingly, a recent study conducted by Markey \textit{et al.} even challenges the correlations of violence and video game play \cite{markey2014}. In this study, the researchers examined how popular game trends compare to real-world crime rates. The results showed that there is no evidence that violent video games are positively correlated to real-world crime rates in the United States. Since causation has not yet been proven, and now even the correlations that incited the urge to regulate are being called into question, many feel it is unfair to censor or otherwise regulate media that has not been shown to be directly responsible for aggressive behavior. 

%%%%%%%%%%-------gov't deciding what is right---------
A similar argument, complementing the case for individual rights, is that the government should not be able to dictate what is morally ``right" and ``wrong" in regards to the media. Firstly, a government's conceptualization of right and wrong may not be similar to the opinions of the citizens. Even if most of the citizens agreed, the issue of individual rights surfaces again, and each person should be able to decide for his or herself what type of media is good for them to personally consume.

\hannah{ The following paragraph may be removed....}
A sentiment that is commonly echoed in anti-censorship and even anti-monitoring groups is intellectual freedom, which is the belief that humans have the right to explore and grow intellectually, regardless of the source of information. Julie Cohen emphasizes this right in her paper on intellectual property management \cite{cohen1995}. [...]

Kantian analysis has shown the act of censoring or restricting video games to be ethically wrong. This result from a workable ethical theory, coupled with the infringement of individual rights and the lack of proof of causation provides a strong argument against the censorship or restriction of video games.


%http://www.pcgames.de/Crysis-PC-119380/News/Cevat-Yerli-ueber-die-so-genannten-Killerspiele-691719/2/