%%Conclusion

In our paper, we did ethical analysis on the topic "Is censorship ethical to be applied to regulate violent online games?" using Kantian theories and utilitarianism for both supporting and being against censorship, respectively. Since more and more experiments and data shows correlations between aggressive behaviors of children and media violence, especially violent online games, it is necessary for everyone to be aware of what is happening now. According to our analysis, Kantian theories proves that censorship is unethical because it violates both 1st and 2nd formulations. It also undermine the freedom of speech which would stifle the creation and imagination of designers. However, from another point of view, utilitarianism indicates the necessity of censorship since the benefits(positive impacts) for all of the stakeholders are far more than the cost(negative impacts). Based on the existing solutions, we present some possible solutions for this ethical issue, in law, market, social norms and code parts, aiming to find a compromise when considering the interests of stakeholders.

In future work, we will still focus on the improvement of solutions since internet and digital media technologies are changing very fast so that  it keeps creating vacuum in regulations. Besides, our proposed solutions are based on our analysis which may face some difficulties when taking into practice. It is necessary to consider the feasibility and effectiveness of them according to the actual situation of our society. Despite that ethical issues involving in the digital media technologies and internet are always open topics, they are urging to be highly emphasized and considered comprehensively. At last, no matter what our world will be changed by technologies, it requires everyone to have this consciousness of self-protection, objectivity, and reality.