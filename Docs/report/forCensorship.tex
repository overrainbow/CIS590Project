%outlines for Utilitarianism
%Correlations between violent video games and aggressive behaviors of youth
%stakeholders and act utilitarianism analysis
\hannah{Since our paper is a debate, we want this section to be strongly for censorship. Because of that, we don't want to point out first thing that there is no solid evidence. Instead, we want to emphasize the correlations between violence and video games.}
\indent\indent Admittedly, for now, there is still lacking of supporting evidence such as theory analysis or persuasive experiments to prove and further establish the causal relationship between violent video games and aggressive behaviors of young generations. However, based on existing experiments and result analysis, \hannah{Remove 'apparently'}apparently there is \cor{are} some correlations between \hannah{these violent games and aggressive and violent behavior} these media violence, and current and even future generations. In other words, \sout{caused by violent video games,} there are some very obvious impacts which may damage communication skills, morality and even personality of our children.

 \hannah{Generally, one should not start a sentence with a citation. Maybe rearrange it to be ``There have been several experiments examining the biological and psychological effects of violent video games \cite{barlett2009, ferguson2008, gentile2004, anderson2003}." }\cite{barlett2009}, \cite{ferguson2008}, \cite{gentile2004}, and \cite{anderson2003} shows different kinds of experiments designed and implemented in biological, psychological and statistical ways. In \cite{gentile2004}, the authors effectively chose varieties of participants at different ages, with different genders, and from different locations(schools, cities, etc.) for their experiments. After data collection and analysis, they did a survey on different individual participants about their behaviors of both at school and in their daily life. Statistics of their media habits indicates that children in 8 and 9 grades are more likely to play video games around 9 hours per week, and males play more than females. As a result, these adolescents usually prefer to spend more time on watching a movie, listening to music, rather than reading or studying. Another result about their interested games is that only 1\% of boys and 16\% of girls prefer to have no violence in video games. Based on these data, the survey shows that 23\% of children reported that they may argue or have conflicts with their teachers ‘‘almost weekly’’ or ‘‘almost daily’’. And 34\% of them reported at least a physical fight within past year. 
 
 In \cite{anderson2003}, several studies on media violence and aggression are discussed. For example, Bjorkqvist's experiment(1985) \hannah{Which paper is the Bjorkqvist experiment from? We should add it to BibTeX and cite it in the same way we cites the others, using the cite command.}  is designed to display two kinds of films, violent and non-violent, in front of randomly chosen 5 to 6-year-old Finnish children in order to observe their further behaviors. The result shows that those who only watched violent films reach a higher rate on physical assaults, as well as other type of aggression. Another experiment from Josephson(1987) \hannah{Again, we should add this to BibTeX} indicates that violent content in the films (such as physical attack in a hockey movie) will remind those boys (7 to 9-year-old) who have watched this movie and stimulate their physical attacks and other aggressive behaviors when they play hockey. Unfortunately as we know, at present, more general proof theories, universal experiment framework, and unified evaluation standards which are used for proving these causal relationships still needs to be discussed and finally established. As a result, it leads to many disagreements and criticisms on these experiments which poses lots of difficulties on drawing a certain conclusion that there exists such causal relationship between digital media violence and aggression behaviors. However, it is not hard to see that there are some very common features including experimental procedures, sampling and data processing methods, etc. among these controversial experiments, such as random sampling, large number of participants, wide range of methods, and so on which to some extent prove the objectivity, fairness and reliability of their results. Therefore, we can conclude that there really exists some correlations between media violence and aggression behaviors of youth even though the actually causal relationship is still expected to be proved. 

In order to prove the necessity of censorship, utilitarianism analysis will be applied and discussed in the remaining part of this section. Our proposition should be censorship is necessary to be applied to constrain both designers and purchasers' behaviors. And now here is the question, is censorship ethical or not? 

According to this issue, consider the stakeholders first. Obviously, without censorship, violent video games may have both positive and negative influences on different social groups, individuals, and official organizations. For example, the negative impact would be that it may decrease the force of laws, disturb the social orders, hinder education, and so on. On the other hand, it can maximize the benefits for manufacturers and designers. Since it is not that easy to directly point out if censorship is ethical or not, here we will choose government, police, hospitals, education organizations, game manufacturers, parents, youth and our whole society as stakeholders to do ethical analysis, and try to reach a conclusion on this problem.

Government has a strong responsibility of maintaining the harmony and safety of society. In order to achieve this goal, it is necessary to control and predict the trends of people's attitudes, opinions, judgments and behaviors. \hannah{I would remove `Apparently,' as it sounds like you are not certain of the issue} Apparently, for violent video games, censorship provides a way to control their impacts by regulating designers' behavior using some unified standards and principles. Since we admit the relations between these violent games and aggressive behaviors of youth, censorship helps a lot in avoiding the potential and unpredictable risks and threats involving intentional attacks, physical assaults, and many other aggressive behaviors stimulated by these media violence. In this way, government can better govern the public, so that their regime can be consolidated, and the force of laws and regulations can be strengthened.

Second, consider other official organizations and social groups, here we set police and hospital as examples. Censorship undoubtedly has positive influences on these groups and organizations. For example, censorship can help to constrain the age of purchasers, such as "this kind of product cannot be sold to the youth under 12-year-old." Also, it regulates the game designers' behaviors like games with too much blood and violence contents cannot come into the market. In this way, it reduces the possibilities of children accessing violent video games so that juvenile delinquency could be effectively decreased and controlled. Consequently, the number of victims, injuries, and harms are also decreasing. For both police departments and hospitals, their workload and stress can be relieved, as well as their supplement requirements and financial pressure.

Another set of stakeholders benefitting from censorship are education organizations, like schools, universities, colleges and some other social education organizations. When educating our children, educators not only aim to teach students new knowledge, but also tell them how to be a ``good" person. ``Good" has a wide range of meanings, here we emphasize on the ethics part. School managers need to monitor and regulate students' behaviors in order to avoid any unnecessary accidents. It is very common that students may have some disagreements or other conflicts with each other, their teachers, or even their parents, and sometimes it is hard to predict and further control their following actions. Luckily, censorship provides a way to keep our children away from violence contents hidden in games which may stimulate their aggressive behaviors. Thanks to censorship, it is easier for educators to monitor, and control students' behaviors, establish education strategies and design educating methods. 

Now we consider the interests of game manufacturers. Most of these code designers create games in order to make profits. In other words, not all of the game makers are altruistic. \hannah{Further explain why some would be altruistic?} As a result, censorship will undoubtedly shrink their benefits. For instance, it limits the range of customers for a certain product based on their age or gender, which will decrease the total amount of sales. Another concern is that censorship may depress their motivations and passions of creating novel games. Creativity plays an important role in Read-Write culture which is the core of the improvement of human civilization, as Lawrence said in his TED talk, too many constraints may pose obstacles to designers' creativity and imaginations which will damage this expected culture. In addition, as mentioned in the previous section, censorship undermines free speech. Democracy needs free speech, the public needs free speech, our society needs free speech. Everyone, no matter who you are, where you belong to, should have equal rights for speech. Especially for those weak groups, like communities of color, religion, and disability, they need the right to speak out loudly, to show their feelings, to tell their own stories and to fight for justice. Since censorship constrains free speech of the public, it will obviously cause some negative impacts on game manufacturers.

Furthermore, censorship is an important tool for parents to protect their children from accessing media violence, and any other ``unhealthy" contents spreading through the internet. Parents usually have less time on observing, monitoring and controlling the actions of their children because of busy working and stress. Another problem is that it is hard for parents getting involved in this issue since they have less awareness and sometimes can hardly better understand the further impact of violent games on their children.

In addition, individual(young players) is one of the most important parts considered in this ethical issue. As we know, because lacking of consciousness of self-discipline and hard to be objective, our children cannot fairly judge and evaluate what they can do and what they cannot. Most of their improper behaviors just depend on curiosity,  a sudden impulse, or the instigation and stimuli from others. It is very common that many adolescents may face difficulties of distinguishing the real world from virtual world--game addiction. This can easily cause some unexpected troubles. For example, a child may have some aggressive behaviors on the other children just because he played a game which reminded him of the similar actions. For our young generation, social norms is also a very critical concern especially in the internet. When playing online games, almost all the designers try their best to build a more interactive virtual world for customers, this means everyone can communicate with others, both in the way of text and voice. As a result, this kind of "free speech" leads to an out of control situations which may misleading our children and destroy their correct morality. These interactions, especially in violent online games, usually contains some ``foul languages", some aggressive emotions and other bad contents which aims to insult, hurt, slander others. These online behaviors are hard to control because of anonymity, so that to some extent they will ``hurt" our children as well. This may stimulate their aggressive behaviors like physical attacks or some other dangerous actions.Therefore, censorship can decrease the risks of bad behaviors, and help our children get away from those violent content. It is undoubtedly good for their growth, and will increase their awareness of self-protection, and sense of responsibilities for  maintaining peace, harmony and safety.

Lastly, censorship is an effective way to keep our society harmonious and safe because it regulate behaviors of related person so that the potential risks and threats of aggressive behaviors(e.g. shooting criminals) can be controlled and reduced.

Based on the discussion above, censorship is necessary for us. However, a more comprehensive and effective solution is still needed since our regulations(code, social norms, laws and market) still have some loopholes which provides those premeditated people a way to spread media violence. For example, there exists some very tiny online games with bloody and passive content which are widely spread and allow free download through the whole internet. What's more, because of anonymity, it cannot be monitor and control in time. Here the imperfection of our regulations is exposed which needs to be further discussed.









