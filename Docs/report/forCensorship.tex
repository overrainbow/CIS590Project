%outlines for Utilitarianism
%Correlations between violent video games and aggressive behaviors of youth
%stakeholders and act utilitarianism analysis
\indent\indent As mentioned in the previous section, we have to admit that there is still a lack of supporting evidence such as theory analysis or persuasive experiments to prove and further establish the causal relationship between violent video games and aggressive behaviors of young generations. However, based on existing experiments and result analysis, there are powerful correlations between these violent games and aggressive and violent behavior. In other words, there are some very obvious impacts of these games which may damage communication skills, morality and even personality of children.

 There have been several experiments examining the biological and psychological effects of violent video games \cite{barlett2009, ferguson2008, gentile2004, anderson2003}. For example, in Gentile's study \cite{gentile2004}, the authors effectively chose varieties of participants at different ages, with different genders, and from different locations (schools, cities, etc.) for their experiments. After data collection and analysis, they did a survey on different individual participants, inquiring about their behaviors of both at school and in their daily life. Statistics of their media habits indicate that children in eigth and ninth grades on average play video games around 9 hours per week, and males play more than females. The study also showed that these adolescents usually prefer to spend more time watching a movie, listening to music, rather than reading or studying. Another interesting result is that only 1\% of boys and 16\% of girls prefer to have no violence in video games, which shows that violence is desired content for most children. The authors' survey also shows that 23\% of children reported that they argue or have conflicts with their teachers almost weekly�� or ��almost daily��. Similarly, 34\% of them reported at least one a physical fight within past year. This demonstrates a correlation of aggressive behavior and violent video game play.
 
In a study by Anderson \textit{et al.}, several studies on media violence and aggression are discussed\cite{anderson2003}. For example, an experiment by Bjorkqvist is designed to display two kinds of films, violent and non-violent, in front of randomly chosen 5 to 6-year-old Finnish children in order to observe their further behaviors \cite{bjorkqvist1985}. The result shows that those who only watched violent films reach a higher rate on physical assaults, as well as other type of aggression than those who watched the non-violent films. Another experiment from Josephson indicates that violent content in films, such as physical attacks in a hockey movie, influenced the 7 to 9-year-old boys who watched them to express more aggressive behavior when they played hockey than was expressed by boys who did not watch the violent films \cite{josephson1987}.

Despite the current lack of proof of causation, the correlations between media violence and aggressive behaviors are clear. Violence in games does have a significant impact on youth and will cause some further influence on future generations, as shown in the above studies and in many more. These studies use well-respected experimental methods, including random sampling, a large number of participants, and a wide range of methods, which prove the objectivity, fairness and reliability of their results. Therefore, we can conclude that there really exists some correlations between media violence and aggression behaviors of youth even though the actually causal relationship is still expected to be proved.    

In order to prove the necessity of censorship, utilitarianism analysis will be applied and discussed in the remaining part of this section. Our proposition is that censorship is necessary to be applied to constrain both designers and purchasers' behaviors. And now here is the question, is censorship ethical or not? 

According to this issue, we consider the stakeholders first. Obviously, without censorship, violent video games may have both positive and negative influences on different social groups, individuals, and official organizations. For example, the negative impact would be that it may decrease the force of laws, disturb the social orders, hinder education, and so on. On the other hand, it can maximize the benefits for manufacturers and designers. Since it is not that easy to directly point out if censorship is ethical or not, here we will choose government, police, hospitals, education organizations, game developers, parents, youth and our whole society as stakeholders to do ethical analysis, and try to reach a conclusion on this problem.

Government has a strong responsibility of maintaining the harmony and safety of our society. In order to achieve this goal, it is necessary to control and predict the trends of people's attitudes, opinions, judgments and behaviors. For violent video games, censorship provides a way to control their impacts by regulating designers' behavior using some unified standards and principles. Since we have shown the relations between these violent games and aggressive behaviors of youth, censorship helps a lot in avoiding the potential and unpredictable risks and threats involving intentional attacks, physical assaults, and many other aggressive behaviors stimulated by these media violence. In this way, government can better govern the public, so that their regime can be consolidated, and the force of laws and regulations can be strengthened.

Second, consider other official organizations and social groups, here we set police and hospital as examples. Censorship and regulation undoubtedly have positive influences on these groups and organizations. For example, regulation can help to constrain the age of purchasers of violent games by creating age ratings on each product. Also, censorship regulates the game designers' behaviors such that games with too much blood and violent content must be censored or they cannot come into the market. In this way, it reduces the possibilities of children accessing violent video games so that juvenile delinquency could be effectively decreased and controlled. Consequently, the number of victims, injuries, and harms are also decreasing. For both police departments and hospitals, their workload and stress can be relieved, as well as the pressure of keeping adequate supplies on hand.

Another set of stakeholders benefitting from censorship are education organizations, like schools, universities, colleges and some other social education organizations. When educating our children, educators not only aim to teach students new knowledge, but also tell them how to be a ``good" person. ``Good" has a wide range of meanings, here we emphasize on the ethics part. School managers need to monitor and regulate students' behaviors in order to avoid any unnecessary accidents. It is very common that students may have some disagreements or other conflicts with each other, their teachers, or even their parents, and sometimes it is hard to predict and further control their following actions. Luckily, censorship provides a way to keep our children away from violence content hidden in games which may stimulate their aggressive behaviors. Thanks to censorship, it is easier for educators to monitor, and control students' behaviors, establish education strategies and design educating methods. 

Now we consider the interests of game manufacturers. 
%Most of these code designers create games to make profits. In other words, not all of the game makers are altruistic. Some developers create and recreate for their own interest or to serve people, but some do not. Everyone deserves a better life and career. It is very common and can be totally understood if a designer wants to earn more money to improve the life condition of his or her family by selling his or her works.
Censorship will undoubtedly shrink their benefits. For instance, it limits the range of customers for a certain product based on their age, which will decrease the total amount of sales. Another concern is that censorship may depress their motivations and passions of creating novel games. Creativity plays an important role in Read-Write culture which is the core of the improvement of human civilization, as Lawrence said in his TED talk. Too many constraints may pose obstacles to designers' creativity and imaginations which will damage this expected culture. In addition, as we know, democracy needs free speech, the public needs free speech, our society needs free speech. It is obvious that censorship undermines the freedom of speech which is also mentioned in the previous section.  

Censorship is, however, an important tool for parents to protect their children from accessing media violence, and any other ``unhealthy" content spreading through the internet. In this busy era, parents usually have less time for observing, monitoring and controlling the actions of their children because of their demanding job and stress. Another problem is that it is hard for parents to get involved in this issue since they have less awareness and knowledge of newer technologies and often do no understand the further impact of violent games on their children. Censorship will be a useful aid for busy parents to protect their children in this rapidly changing world.

Young players are one of the most important stakeholders considered in this ethical issue. As we know, the minds of children are still developing and as a result, they have a reduced ability to judge moral rights and wrongs, be self-disciplined, and understand the line between fantasy and reality. Curiosity, a sudden impulse, or pressure and stimuli from others often cause a child to engage in improper behaviors. It is very common that many adolescents may face difficulties of distinguishing the real world from virtual world and easily suffer from game addiction. This can easily cause issues. For example, a child may engage in aggressive behaviors with other children just because he played a game which reminded him of the similar actions. For our young generation, social norms is also a very critical concern especially in the internet. When playing online games, children can interact verbally or in text with other players, who are often much older. These interactions, especially in violent online games, usually contain foul language, aggressive reactions and other bad content which aims to insult, hurt, slander others. Since children often learn by mimicry, it is very common to hear children repeat rude and disrespectful language heard in games. Therefore, censorship can decrease the risks of bad behaviors, and protect children from violent content that may cause them to act out aggressively. It is undoubtedly good for their growth, and will increase their awareness of self-protection, and sense of responsibilities for maintaining peace, harmony and safety.

Lastly, censorship is an effective way to keep our society harmonious and safe because it reduces the negative influences on behavior so that the potential risks and threats of aggressive behaviors, like shootings and fights, can be reduced.

Based on the discussion above, censorship or regulation of violent games is ethical and beneficial since it brings more benefits than cost. However, currently a more comprehensive and effective solution is still needed since our regulations (code, social norms, laws and market) still have some loopholes which provide people a way to spread media violence. For instance, since the Internet is global and allows anyone to upload what they wish, developers can make games with bloody and passive content freely available to anyone with an Internet connection. Additionally, because of anonymity and the fact that everything is just bits, it is difficult to monitor and control the distribution of such games. Here the imperfection of our regulations are exposed which needs to be further discussed.









